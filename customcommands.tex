\newcommand{\rectfrompagetop}[2]{%
		\begin{tikzpicture}[remember picture,overlay]%
			\node at (current page.north west) {%
				\begin{tikzpicture}[remember picture, overlay]%
					\fill[#1] (0,0) rectangle ++(\paperwidth, -#2);%
				\end{tikzpicture}%
			};%
		\end{tikzpicture}%
}%

\newcommand{\rectfrompagebottom}[2]{%
		\begin{tikzpicture}[remember picture,overlay]%
			\node at (current page.south west) {%
				\begin{tikzpicture}[remember picture, overlay]%
					\fill[#1] (0,0) rectangle ++(\paperwidth, #2);%
				\end{tikzpicture}%
			};%
		\end{tikzpicture}%
}%

\newcommand{\keyword}[2][neonsecondary]{\@neonheadingfont\textcolor{#1}{#2}\normalfont}
\newcommand{\kw}{\keyword}

\newcommand{\circuit}[7][NOCAPTION]{ % circuitikz circuit
	\fig[#1]{#2}{#3}{#4}{#5}{\begin{circuitikz}[scale=#6, transform shape]
		#7
	\end{circuitikz}
		}
}
\newcommand{\fcircuit}[7][NOCAPTION]{ % file circuit
	\circuit[#1]{#2}{#3}{#4}{#5}{#6}{
	\input{res/circ/#7}
	}
}

\newcommand{\mpni}[1]{
	\noindent\begin{minipage}{\textwidth}
	#1
	\end{minipage}
}

\newcommand{\definitiontitle}[1]{
	{
		\@neonheadingfont%
		\small%
		\bfseries%
		\color{neonsecondary}%
		#1%
	}
}

\newcommand{\goal}[2][0em]{
	\begin{adjustwidth}{1em}{0cm}
		\@neonheadingfont%
		\small%
		\color{neonprimary}%
		#2%
		\vspace{#1}%
	\end{adjustwidth}
}

\newcommand{\definition}[2]{
	\marginpar{
		\RaggedRight%
		\definitiontitle{#1}\\
		\footnotesize #2%
		\vspace{1em}%
	}
}

\newcommand{\kwdef}[3][_]{
	\kw{#2}%
	\ifstrequal{#1}{_}{%
		\definition{#2}{#3}%
	}{%
		\definition{#1}{#3}%
	}%
}

\newcommand{\sectionref}[1]{%
	{\autoref{#1} \textit{\nameref{#1}}}%
}%

\newcommand{\todo}[1]{
	\color{red} {\@neonheadingfont todo} #1
	\begin{tikzpicture}[remember picture,overlay]
			\node at (current page.south west) {
				\begin{tikzpicture}[remember picture, overlay]
					\fill[neonprimary] (0,0) rectangle ++(1cm, \paperheight);
					\node[rotate=90, anchor=west] at (0.5cm,0.5cm) {\@neonheadingfont \color{white} Entwurf};
				\end{tikzpicture}
			};
	\end{tikzpicture}
}

% Wrapfigure commands from https://tex.stackexchange.com/questions/317447/wrapfig-and-wraptable-not-obeying-floatbarrier
\def\WFfill{%
	\ifx\parshape\WF@fudgeparshape %
	\nobreak  %
	\ifnum\c@WF@wrappedlines>\@ne %
	\advance\c@WF@wrappedlines\m@ne % 
	\vskip\c@WF@wrappedlines\baselineskip %
	\global\c@WF@wrappedlines\z@ %
	\fi %
	\allowbreak %
	\WF@finale %
	\fi %
} 

\let\OldFloatBarrier\FloatBarrier
\renewcommand{\FloatBarrier}{\WFfill\WFclear\OldFloatBarrier}
\newcommand{\FloatBarrierCorrect}{\vspace{-2.5em}}

\newcommand{\goaldef}[2]{
	\csgdef{goaldef#1}{#2}
	\hyperref[goaldef:#1]{#2}
}

\newcommand{\goalrefh}[1]{
	\label{goaldef:#1}
}

\newcommand{\goalref}[2][0em]{
	\label{goaldef:#2}
	\goal[#1]{\csuse{goaldef#2}}
}

\newcommand{\goalcp}[2][0em]{
	\goal[#1]{\csuse{goaldef#2}}
}

\newcommand{\goallink}[2]{%
	\hyperref[goaldef:#1]{#2}%
}

\newcommand{\goalsref}[1]{%
	\sectionref{goaldef:#1}%
}